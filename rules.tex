\documentclass[a4paper,12pt]{article}

\usepackage[utf8]{inputenc}
\usepackage[T2A]{fontenc}
\usepackage[english,russian]{babel}

\usepackage{graphicx}
\usepackage{color}
\usepackage{indentfirst}
\usepackage{fancyhdr}
\usepackage{needspace}

\pagestyle{fancy}
\lhead{Настолка пока что без названия}
\rhead{ЛКШ}
\rfoot{2015, август}

\begin{document}
  \section{Правила}

    \subsection{Легенда}

    \subsection{Подготовка игры}

    \subsection{Ход игры}

      Игра состоит из нескольких ходов, выполняемых по очереди.

      Ход одного игрока состоит из следующих действий:
      \begin{enumerate}
        \item Взять очередную карту из колоды событий.
          Все игроки, начиная с ходящего, выполняют коллективное
          действие, описанное на ней. Карта отправляется в колоду
          сброса.
        \item Взять себе в руку взакрытую очередную карту из колоды.
        \item Можно сыграть одну или несколько карт из руки.
        \item Можно произвести одно или несколько исследований.
      \end{enumerate}

    \subsection{Завершение игры}

      Игра завершается, когда один из игроков на своем ходу
      открывает портал. % изобретает трактор

    \subsection{Карты событий}

      Каждая карта включает в себя:
      \begin{itemize}
        \item Название
        \item Легенду
        \item Действие на всех игроков
        \item Действие на одного игрока
        \item Стоимость действия на одного игрока
      \end{itemize}

    \subsection{Характеристики игроков}

    \subsection{Расы}

      Эльфы, гномы, хоббиты, гоблины, орки, летучие каракатицы.

    \subsection{Исследования}
\end{document}
